
    \documentclass[12pt]{article}
    \usepackage{amsfonts,amsmath,amssymb}
    \usepackage{amsmath,multicol,eso-pic}
    \usepackage[utf8]{inputenc}
    \usepackage[T1]{fontenc}
    \usepackage[left=2.00cm, right=2.00cm, top=2.00cm, bottom=2.00cm]{geometry}
    \usepackage{titlesec}
    \usepackage{enumerate}
    \usepackage{breqn}
    \usepackage{tikz}
    \usepackage{rotating}
    %\renewcommand{\wedge}{~}
    %\renewcommand{\neg}{\overline}
    \titleformat{\section}{\large}{\thesection.}{1em}{}
    \usepackage{mathtools}
    \DeclarePairedDelimiter{\floor}{\lfloor}{\rfloor}
    % % % % % POPUNITE PODATKE
    
    \newcommand{\prezimeIme}{Mašović Haris}
    \newcommand{\brIndexa}{17993}
    \newcommand{\brZadace}{2}
    
    % % % % % 
    
    \begin{document}
    
    \thispagestyle{empty}
    \begin{center}
      \vspace*{1cm}

      \vspace*{2cm}
      {\huge \bf Zadaća \brZadace } \\
      \vspace*{1cm}
      {\Large \bf iz predmeta Diskretna Matematika}

      \vspace*{2cm}

      {\Large Prezime i ime: \prezimeIme} \\
      \vspace*{0.75cm}
      {\Large Br. indexa: \brIndexa} \\
      \vspace*{0.75cm}
      {\Large Demonstrator: Rijad Muminović} \\
      \vspace*{0.75cm}
      {\Large Grupa: RI - 4} \\

%%      \vspace*{3cm}
%      \renewcommand{\arraystretch}{1.75}
 %     \begin{tabular}{|c|c|}
  %  	\hline Zadatak & Bodovi \\
   % 	\hline 1 &  \\
%    	\hline 2 &  \\
 %   	\hline 3 &  \\
  %  	\hline 4 &  \\
  %  	\hline 5 &  \\
   % 	\hline 6 &  \\
    %	\hline
%     \end{tabular} 

      \vfill


      {\large Elektrotehnički fakultet Sarajevo, Novembar 2017g.}

    \end{center}
    \newpage
    \thispagestyle{empty}
    
    
    % % % % % Rješenja zadataka
	\begin{enumerate}
		\item Rješenje zadatka \\
		\\
		Zadatak 1 [0.25 poena] \\
		
Odredite na koliko je različitih načina moguće razmjestiti 9 studenata i 4
profesora oko okruglog stola ako je poznato da se profesori međusobno ne
podnose i ne mogu sjediti jedan do drugog. \\

* S obzirom da studente redamo oko okruglog stola, radi se o kružnim
permutacijama, a svaka permutacija daje drugi raspored. Kao što znamo,
broj kružnih permutacija za n elemenata je (n-1)! . Dakle studente možemo
rasporediti na $P_{9}$ = (9 - 1)! = 8! = 40320 načina. Preostalo nam je
još da razmjestimo 4 profesora, što je druga etapa ovog problema. Kako svi
sjede za okruglim stolom, a nikoja dva profesora ne mogu sjediti jedan do
drugog, njih ćemo razmjestiti tako da sjede između studenata. Prvog od
četiri profesora možemo staviti na jedno od 9 mogućih mjesta (između dva
studenta), drugog profesora možemo staviti na jedno od 8 preostalih mjesta
(jedno smo već zauzeli), trećeg na jedno od 7 preostalih mjesta (dva su već
zauzeta), četvrtog na jedno od 6 preostalih mjesta (tri su već zauzeta).
Dakle profesore možemo razmjestiti na 9 ${\cdot}$ 8 ${\cdot}$ 7 ${\cdot}$ 6 = 3024 načina. 
Konačno na osnovu multiplikativnog
principa, ukupan broj načina na koji možemo izvršiti razmještanje je
\begin{equation*}
    P_{9} \cdot 3024 = 40320 \cdot 3024 = 121927680.
\end{equation*}
        
		\item Rješenje zadatka \\
		\\
		Zadatak 2 [0.25 poena] \\
		
Potrebno je formirati osmočlanu ekipu za međunarodno softverskohardversko
takmičenje. Uvjeti su da ekipa mora imati barem četiri studenta sa
smjera RI, dok su studenti drugih smjerova poželjni (zbog većeg hardverskog
znanja) ali ne i obavezni. Za takmičenje se prijavilo 9 studenata smjera RI i 6
studenata smjera AiE (dok studenti drugih smjerova nisu bili zainteresirani).
Odredite na koliko načina je moguće odabrati traženu ekipu. Koliko će iznositi
broj mogućih ekipa ukoliko se postavi dodatno ograničenje da ekipa mora
imati i barem jednog studenta smjera AiE? \\
		
* Da bi formirali ekipu, potrebno je izabrati k studenata sa smjera RI i
8 - k studenata sa smjera AiE, pri čemu je k ${\geq}$ 4 po uvjetu zadatka. Kako
mora vrijediti k ${\geq}$ 0 i 8 - k ${\geq}$ 0, dobijamo da je k ${\in}$ $\{$ 4, 5, 6, 7, 8$\}$. 
Pošto se svi studenti razlikuju, za neko k studente sa smjera RI možemo izabrati na C(9, k) načina, 
dok 8 - k studenata smjera AiE možemo izabrati na C(6, 8 - k) načina. 
Kako su oba izbora neovisna jedan od drugog, za broj načina za izbor ekipe koristiti 
ćemo multiplikativni princip za neko k i on iznosi		
\begin{equation*}
    C(9, k) \cdot C(6, 8 - k)
\end{equation*}
Na kraju korištenjem aditivnog principa rješenje problema dobijamo sumi-
ranjem za sve vrijednosti k iz skupa, pa je konačna formula		
\begin{equation*}
   \sum_{k=4}^8 ~ C(9, k) \cdot C(6, 8 - k)
\end{equation*}		

\newpage
odnosno
\begin{equation*}
   C(9, 4) \cdot C(6, 4) + C(9, 5) \cdot C(6, 3) + C(9, 6) \cdot C(6, 2) + C(9, 7) \cdot C(6, 1)
   + C(9, 8) \cdot C(6, 0) =
\end{equation*}	
\begin{equation*}
   126 \cdot 15 + 126 \cdot 20 + 84 \cdot 15 + 36 \cdot 6 + C(9, 8) \cdot C(6, 0) = 5886 + C(9, 8) \cdot C(6, 0)
\end{equation*}	
\begin{equation*}
   5886 + 9 = 5895
\end{equation*}
Vidimo da traženu ekipu možemo odabrati na 5895 načina. Dalje se postavlja uslov koliko će iznositi broj mogućih ekipa ukoliko se postavi dodatno ograničenje da ekipa mora imati i barem jednog studenta smjera AiE.
U već izračunatoj vrijednosti sume k = 8 se ne smije pojavljivati, samim tim ukupan broj mogućih kombinacija da se izabere ekipa sa barem jednim studentom AiE je 5886.

		\item Rješenje zadatka \\
		\\ 
		Zadatak 3 [0.25 poena] \\ 
		
Odredite na koliko načina možemo raspodijeliti 6 jabuka, 12 smokava i 8
naranči među osmero djece, pri čemu se pretpostavlja se da se primjerci iste
voćke ne mogu međusobno razlikovati (npr. sve jabuke su iste). \\
		
* Ovu raspodjelu možemo izvršiti tako da prvo raspodijelimo jabuke, pa
smokve pa onda naranče, a konačan broj načina raspodjele dobijamo multiplikativnim 
principom jer su sve ove raspodjele neovisne jedna od druge. S
obzirom da se voćke međusobno ne razlikuju, a problem raspodjele svake od
njih je raspodjeljivanje n jednakih elemenata među k ljudi možemo zaključiti
da se radi o kombinacijama sa ponavljanjem
\begin{equation*}
   \overline{C}(k, n) = C(k, n+k-1) =  {{n+k-1}\choose{k}}
\end{equation*}
Izračunajmo sada:
\begin{equation*}
   \overline{C}_{~8}^{~6} \cdot \overline{C}_{~8}^{~12} \cdot \overline{C}_{~8}^{~8} = 
\end{equation*}
\begin{equation*}
   = C_{~13}^{~6} \cdot C_{~19}^{~12} \cdot C_{~15}^{~8} =
\end{equation*}
\begin{equation*}
   = {{13}\choose{6}} \cdot {{19}\choose{12}} \cdot {{15}\choose{8}} = 1716 \cdot 50388 \cdot 6435 = 556407474480
\end{equation*}

		\item Rješenje zadatka \\
		\\
		Zadatak 4 [0.25 poena] \\
		
Na stolu se nalazi određena količina papirića, pri čemu se na svakom od
papirića nalazi po jedno slovo. Na 2 papirića se nalazi slovo A, na 2 papirića se
nalazi slovo M, na 3 papirića slovo Y i na 2 papirića slovo V. Odredite koliko se
različitih troslovnih riječi može napisati slažući uzete papiriće jedan do drugog
(nebitno je imaju li te riječi smisla ili ne).

        \newpage
* Imamo 2 slova A, 2 slova M, 3 slova Y i 2 slova V. Pošto pravimo troslovne
riječi, tražimo koeficjent uz $t^3$, od tih slova znači da nam je poredak odnosno redoslijed uzimanja tih
slova bitan, što nas dovodi do zaključka da se radi o permutacijama sa ponavljanjem 
klase 3. Za rješavanje ovog zadatka koristiti ćemo funkcije izvodnice.
\begin{equation*}
   \Psi_{n;m_1,m_2,...,m_n} = \prod_{i=1}^{n} ~ \sum_{j=0}^{m_i} \frac{t^j}{j!}
\end{equation*}
Razvijajući ovaj polinom za n = 4, $m_1$ = 2, $m_2$ = 2, $m_3$ = 3, $m_4$ = 2
tražimo član uz $t^3$. Ako ovaj koefcijent označimo sa ${\lambda_3}$ konačno rješenje ovog
zadatka će biti 3! ${\cdot}$ ${\lambda_3}$ 
\begin{equation*}
   \Psi_{4;2,2,4,2} = (1+t+\frac{t^2}{2})^{3} \cdot (1+t+ \frac{t^2}{2}+ \frac{t^3}{6})
\end{equation*}
odnosno član uz $t^3$:

\begin{equation*}
   \Psi_{4;2,2,4,2} = 1 + 4 \cdot t + 8 \cdot t^2 + 61 \cdot \frac{t^3}{6} + ...
\end{equation*}
Vidimo da je ${\lambda_3}$ = {$\frac{61}{6}$}. Pomnožimo taj broj sa 3! , i imamo naš broj varijacija sa ponavljanjem koji je 61.
        
		\item Rješenje zadatka \\
		\\
		Zadatak 5 [0.25 poena] \\
		
Odredite koliko se različitih paketa koji sadrže 6 voćki može napraviti ukoliko
nam je raspolaganju 6 breskvi, 3 jabuke, 2 naranče, 3 kruške i 1 smokva (pri
čemu se pretpostavlja da ne pravimo razliku između primjeraka iste voćke). \\

* Ovaj problem je veoma sličan prethodno riješenom problemu u zadatku
4, ali ovdje nije bitan redoslijed kojim stavljamo pojedine voćke u pakete,
tako da ovdje tražimo kombinacije sa ponavljanjem klase 6. Ponovo ćemo
koristiti funkcije izvodnice za računanje.
\begin{equation*}
   \varphi_{n;m_1,m_2,...,m_n} = \prod_{i=1}^{n} ~ \sum_{j=0}^{m_i} t^j
\end{equation*}	
Kada razvijemo ovaj polinom, konačno rješenje će biti koefijent uz član
$t^6$.
\begin{equation*}
   \varphi_{5;6,3,2,3,1} = (1+t+t^2+t^3+t^4+t^5+t^6) \cdot (1+t+t^2+t^3)^2 \cdot (1+t+t^2) \cdot (1+t) =
\end{equation*}	
\begin{equation*}
    = ... + 82 \cdot t^6 + 67 \cdot t^5 + 48 \cdot t^4 + 29 \cdot t^3 + 14 \cdot t^2 + 5 \cdot t + 1
\end{equation*}	
Koefcijent uz $t^6$ je 82, pa je konačno rješenje ovog zadatka 82.
		\newpage
		\item Rješenje zadatka \\
		\\
		Zadatak 6 [0.25 poena] \\
		
Odredite na koliko načina se može rasporediti 33 identičnih kuglica u 6
različitih kutija, ali tako da u svakoj kutiji bude najviše 7 kuglica. \\

* 33 puta biramo jednu od 6 različitih kutija, pri čemu istu kutiju možemo
izabrati maksimalno 7 puta. Dakle računamo kombinacije sa ponavljanjem
klase 33 skupa od 6 elemenata, u ovom slučaju kutija. Za računanje ćemo
koristiti funkcije izvodnice.
\begin{equation*}
   \varphi_{n;m_1,m_2,...,m_n} = \prod_{i=1}^{6} ~ \sum_{j=0}^{7} t^j
\end{equation*}	
Kada razvijemo ovaj polinom, konačno rješenje će biti koefcijent uz član
$t^{33}$. 
\begin{equation*}
   \varphi_{6;n_1,n_2,n_3,n_4,n_5,n_6} = (1+t+t^2+t^3+t^4+t^5+t^6+t^7)^6 =
\end{equation*}	
\begin{equation*}
    = ... + 1966 \cdot t^{33} + 2877 \cdot t^{32} + 4032 \cdot t^{31} + ... + 56 \cdot t^3 + 21 \cdot t^2 + 6 \cdot t + 1
\end{equation*}	
Rješenje ovog zadatka je koefcijent uz $t^{33}$, a to je 1966. \\

		\item Rješenje zadatka \\
		\\
		Zadatak 7 [0.25 poena] \\
		
Odredite na koliko načina se 14 različitih predmeta upakovati u 6 identičnih
vreća (koje nemaju nikakav identitet po kojem bi se mogle razlikovati), pri
čemu se dopušta i da neke od vreća ostanu prazne. \\

* Rješenje ovog zadatka je suma Stirlingovih brojeva druge vrste $S^k_{14}$, pri
čemu k može uzimati vrijednosti od 0 do 6. Stirlingovi brojevi 2 vrste se računaju formulom
\begin{equation*}
S^k_n = S^{k-1}_{n-1} + k \cdot S^k_{n-1}
\end{equation*}
pri početnim vrijednostima
\begin{equation*}
S^k_n = 0 \quad \forall ~ n < k
\end{equation*}
\begin{equation*}
S^0_n = 0 \quad \forall n > 0 
\end{equation*}
\begin{equation*}
S^0_0 = 1 
\end{equation*}
Stirlingove brojeve 2 vrste najlakše možemo izračunati pomoću tabele, što ćemo i uraditi. \\
\newpage

\begin{tabular}{|c|c|c|c|c|c|c|c|}
\hline n k &  0&  1&  2&  3&  4& 5 & 6\\ 
\hline 0 &  1&  0&  0&  0&  0& 0&0\\
\hline 1 &  0&  1&  0&  0&  0& 0&0\\
\hline 2 &  0&  1& 1&  0&  0& 0&0\\ 
\hline 3 &  0&  1&  3&  1& 0 & 0&0\\
\hline 4 &  0&  1&  7&  6&  1& 0&0\\
\hline 5 &  0&  1&  15&  25&  10& 1&0\\
\hline 6 &  0&  1&  31&  90&  65& 15&1\\
\hline 7 &  0&  1&  63&  301&  350& 140&21\\
\hline 8 &  0&  1&  127&  966&  1701& 1050&266\\
\hline 9 &  0&  1&  255&  3025&  7770& 6951&2646\\
\hline 10&  0&  1&  511&  9330&  34105& 42525&22827\\
\hline 11&  0&  1&  1023&  28501&  145750& 246730&179487\\
\hline 12&  0&  1&  2047&  86526&  611501& 1379400&1323652\\
\hline 13&  0&  1&  4095&  261625&  2532530& 7508501&9321312\\
\hline 14&  0&  1&  8191&  788970&  10391745& 40075035&63436373\\
\hline
\end{tabular} \\
\\
Kao što smo rekli, rješenje će biti suma svih $S^k_{14}$, tj. u ovom slučaju suma
svih brojeva iz zadnjeg reda ove tabele, pa imamo
\begin{equation*}
   \sum_{k=0}^6 S^k_{14} = 0 + 1 + 8191 + 788970 +10391745 + 40075035 + 63436373 = 114700315
\end{equation*}	

		\item Rješenje zadatka \\
		Zadatak 8 [0.25 poena] \\
		
Odredite na koliko se načina može 14 kamenčića razvrstati u 3 gomilica. Pri
tome se i kamenčići i gomilice smatraju identičnim (odnosno ni kamenčići ni
gomilice nemaju nikakav identitet po kojem bi se mogli razlikovati). \\

* Pošto poredak kod kamenčića i kod gomilica nije bitan, a pri tome gomilice su skupovi
koji ne smiju biti prazni, rješenje ovog zadatka se svodi na računanje broja particija broja 14
sa tačno 3 sabirka odnosno $p^{3}_{14}$. Možemo primjetiti da je naš broj sabiraka mali, tačnije 3, 
pa možemo iskoristiti formulu Honsbergera i Schwennicke-a (Predavanje 5, strana 5), odnosno:
\begin{equation*}
p^3_{14} = \floor[\Bigg]{\frac{14^2}{12} + \frac{1}{2}} = 16.
\end{equation*}
što ujedno predstavlja i rješenje našeg zadatka.
		
		\item Rješenje zadatka \\
		\\
		Zadatak 9 [0.25 poena] \\
		
Odredite na koliko načina se broj 13 može rastaviti na sabirke koji su prirodni
brojevi, pri čemu njihov poredak nije bitan, ali pod dodatnim uvjetom da se
sabirak 3 smije pojaviti najviše 3 puta, dok se sabirak 1 smije pojaviti samo
neparan broj puta. \\ 

		\newpage

* Moramo odrediti broj particija broja 13, ali pošto imamo dodatne uvjete
za odeđene sabirke, najpogodnije je koristiti funkcije izvodnice.
\begin{equation*}
   \varphi_{n;m_1,m_2,...,m_n} = \prod_{i=1}^{k} ~ \sum_{j=0}^{\infty} t^{i \cdot j}
\end{equation*}
Razvijmo ovaj polinom po postavci zadatka:
\begin{equation*}
   \varphi = (t+t^3+t^5+t^7+t^9+t^{11}+t^{13}) \cdot (1+t^2+t^4+t^6+t^8+t^{10}+t^{12}) \cdot (1+t^3+t^6+t^9) \cdot (1+t^4+t^8+t^{12}) \cdot (1+t^5+t^{10})
\end{equation*}
\begin{equation*}
   \cdot (1+t^6+t^{12}) \cdot (1+t^7) \cdot (1+t^8) \cdot (1+t^9) \cdot (1+t^{10}) \cdot (1+t^{11}) \cdot (1+t^{12}) \cdot (1+t^{13})
\end{equation*}
Kad izmnožimo sve imamo:
\begin{equation*}
   \varphi = ... + 25 \cdot t^{14} + 47 \cdot t^{13} + 29 \cdot t^{12} + ... + t^4 + 2 \cdot t^3 + t
\end{equation*}
Kad smo razvili polinom, ostaje nam samo da očitamo koefcijent uz $t^{13}$,
a to je ovdje 47 , što znači da broj 13 možemo rastaviti na prirodne sabirke
na 47 načina ako se sabirak 3 smije pojaviti najviše 3 puta, dok se sabirak 1 smije pojaviti samo
neparan broj puta. \\

		\item Rješenje zadatka \\
		\\
		Zadatak 10 [0.25 poena] \\
		
U nekoj kutiji nalazi se 85 kompakt diskova (CD-ova), od kojih je 18 diskova
nečitljivo. Ukoliko nasumice izaberemo 7 diskova iz kutije, nađite
vjerovatnoću da će \\
a. svi izabrani diskovi biti čitljivi; \\
b. tačno jedan izabrani disk biti nečitljiv; \\
c. barem jedan izabrani disk biti nečitljiv; \\
d. tačno dva izabrana diska biti nečitljiva; \\
e. barem dva izabrana diska biti nečitljiva; \\
f. najviše dva izabrana diska biti nečitljiva; \\
g. najviše dva izabrana diska biti čitljiva; \\
h. svi izabrani diskovi biti nečitljivi. \\

        * \\
        a)
        \begin{equation*}
        \frac{C^7_{67}}{C^7_{85}} = \frac{869648208}{4935847320} = 0.176 = 17.6\%
        \end{equation*}
        b)
        \begin{equation*}
        \frac{C^1_{18} \cdot C^6_{67}}{C^7_{85}} = \frac{18 \cdot 99795696}{4935847320} = 0.363 = 36.3\%
        \end{equation*}
		\newpage
		c)
        \begin{equation*}
        1 - \frac{C^7_{67}}{C^7_{85}} = 1 - \frac{869648208}{4935847320} = 1 -  0.176 = 0.824 = 82.4\%
        \end{equation*}
		d)
        \begin{equation*}
        \frac{C^2_{18} \cdot C^5_{67}}{C^7_{85}} = \frac{153 \cdot 9657648}{4935847320} = 0.299 = 29.9\%
        \end{equation*}
		e)
        \begin{equation*}
        1 - \frac{C^7_{67}+ C^1_{18} \cdot C^6_{67}}{C^7_{85}} = 1 - \frac{869648208 + 18 \cdot 99795696}{4935847320} = 0.459 = 45.9\%
        \end{equation*}
        f)
        \begin{equation*}
        \frac{C^7_{67}+ C^1_{18} \cdot C^6_{67} + C^2_{18} \cdot C^5_{67}}{C^7_{85}} = \frac{869648208 + 18 \cdot 99795696 + 153 \cdot 9657648}{4935847320} = 0.839 = 83.9\%
        \end{equation*}
        g)
        \begin{equation*}
        \frac{C^7_{18} + C^6_{18} \cdot C^1_{67} + C^5_{18} \cdot C^2_{67} }{C^7_{85}} = \frac{31824 + 18564 \cdot 67 + 8568 \cdot 2211}{4935847320} = 0.0040 = 0.4\%
        \end{equation*}
        h)
        \begin{equation*}
        \frac{C^7_{18}}{C^7_{85}} = \frac{31824}{4935847320} = 6.447 \cdot 10^{-6} = 6.447 \cdot 10^{-4}~\%
        \end{equation*}
		\item Rješenje zadatka \\
		\\
		Zadatak 11 [0.25 poena] \\
		\\
Neka je dat pravičan novčić, tj. novčić kod kojeg je jednaka vjerovatnoća
pojave glave ili pisma prilikom bacanja. Ako bacimo takav novčić 60 puta,
očekujemo da će otprilike 30 puta pasti glava i isto toliko puta pismo.
Međutim, to naravno ne znači da će sigurno biti tačno 30 pojava glave ili pisma
(štaviše, vjerovatnoća da se tačno to desi je prilično mala). Odredite: \\
a. Vjerovatnoću da će se zaista pojaviti 30 puta glava i 30 puta pismo; \\
b. Vjerovatnoću da će se glava pojaviti više od 26 a manje od 34 puta; \\
c. Vjerovatnoću da će se glava pojaviti više od 23 a manje od 37 puta. \\
\\
* \\
\\
a) Sa P(A) ćemo označiti vjerovatnoću da je pismo, pošto je novčić
potpuno pravičan, 1 - P(A) je vjerovatnoća da je glava. Vjerovatnoća
za a) se računa kao:
\begin{equation*}
        C^{30}_{60} \cdot (\frac{1}{2})^{30} \cdot (\frac{1}{2})^{30} = 118264581564861424 \cdot (\frac{1}{2})^{60}
        =0.1026 = 10.26~\%
\end{equation*}	
\newpage
b) Ova vjerovatnoća je jednaka sumi vjerovatnoća da će se glava pojaviti
27,28,29,30,31,32,33 puta. Pa imamo:
\begin{equation*}
        C^{27}_{60} \cdot (\frac{1}{2})^{27} \cdot (\frac{1}{2})^{33} +
        C^{28}_{60} \cdot (\frac{1}{2})^{28} \cdot (\frac{1}{2})^{32} +
        C^{29}_{60} \cdot (\frac{1}{2})^{29} \cdot (\frac{1}{2})^{31} +
\end{equation*}	
\begin{equation*}
        C^{30}_{60} \cdot (\frac{1}{2})^{30} \cdot (\frac{1}{2})^{30} +
        C^{31}_{60} \cdot (\frac{1}{2})^{31} \cdot (\frac{1}{2})^{29} +
        C^{32}_{60} \cdot (\frac{1}{2})^{32} \cdot (\frac{1}{2})^{28} +
        C^{33}_{60} \cdot (\frac{1}{2})^{33} \cdot (\frac{1}{2})^{27} 
\end{equation*}	
odnosno:
\begin{equation*}
        2 \cdot C^{27}_{60} \cdot (\frac{1}{2})^{27} \cdot (\frac{1}{2})^{33} +
        2 \cdot C^{28}_{60} \cdot (\frac{1}{2})^{28} \cdot (\frac{1}{2})^{32} +
        2 \cdot C^{29}_{60} \cdot (\frac{1}{2})^{29} \cdot (\frac{1}{2})^{31} +
        C^{30}_{60} \cdot (\frac{1}{2})^{30} \cdot (\frac{1}{2})^{30}
\end{equation*}	
odnosno:
\begin{equation*}
        0.1526 + 0.1799 + 0.1985 + 0.1026 = 0.6336 = 63.36~\%
\end{equation*}	
c) Ova vjerovatnoća je jednaka sumi vjerovatnoća da će se glava pojaviti 24,25,26,34,35,36 puta, odnoso ove vrijednosti ćemo dodati na već sračunate pod b. Pa imamo:
\begin{equation*}
        2 \cdot C^{24}_{60} \cdot (\frac{1}{2})^{24} \cdot (\frac{1}{2})^{36} +
        2 \cdot C^{25}_{60} \cdot (\frac{1}{2})^{25} \cdot (\frac{1}{2})^{35} +
        2 \cdot C^{26}_{60} \cdot (\frac{1}{2})^{26} \cdot (\frac{1}{2})^{34} + b)
\end{equation*}	
odnosno:
\begin{equation*}
         0.0625 + 0.0905 + 0.1212 + 0.6336 = 0.9078 = 90.78~\%
\end{equation*}	
		\item Rješenje zadatka \\
		\\
		Zadatak 12 [0.25 poena] \\
		\\
Odredite vjerovatnoću da će u skupini od 6 nasumično izvučenih karata iz
dobro izmješanog špila od 52 karte dvije karte biti sa slikom i tri karte crvene
boje (herc ili karo). \\
 \\
 * Tražene mogućnosti nisu disjunktne. Napravimo klase koje su disjunktne. \\
 Ukupan broj karata, n = 52 \\
$A_1$ - karte sa slikom crvene boje, $n_1$ = 6 \\
$A_2$ - karte sa slikom koje nisu crvene, $n_2$ = 6 \\
$A_3$ - karte crvene boje koje nisu sa slikom, $n_3$ = 20 \\
$A_4$ - ostale karte, $n_4$ = 20 \\
\\
Iz $A_1$ uzmemo $m_1$, iz $A_2$ $m_2$, iz $A_3$ $m_3$, iz $A_4$ $m_4$.
Sada imamo 6 izvlačenja koje moraju zadovoljiti sljedeći sistem:
\begin{equation*}
    m_1 + m_2 + m_3 + m_4 = 6 \quad
    m_1 + m_2 = 2 \quad 
    m_1 + m_3 = 3
\end{equation*}
\begin{equation*}
    m_1 = 0 \quad m_2 = 2 \quad m_3 = 3 \quad m_4 = 1
\end{equation*}
\begin{equation*}
    m_1 = 1 \quad m_2 = 1 \quad m_3 = 2 \quad m_4 = 2
\end{equation*}
\begin{equation*}
    m_1 = 2 \quad m_2 = 0 \quad m_3 = 1 \quad m_4 = 3
\end{equation*}

Ove mogućnosti se međusobno isključuju, tako da ukupnu vjerovatnoću dobijamo sabiranjem
vjerovatnoća za svaku od ove mogućnosti odnosno:
\newpage
\begin{equation*}
    \frac{C^0_{6} \cdot C^2_{6} \cdot C^3_{20} \cdot C^1_{20} +
          C^1_{6} \cdot C^1_{6} \cdot C^2_{20} \cdot C^2_{20} +
          C^2_{6} \cdot C^0_{6} \cdot C^1_{20} \cdot C^3_{20} 
    }{C^6_{52}} = 
\end{equation*}

\begin{equation*}
    = \frac{1 \cdot 15 \cdot 1140 \cdot 20 + 6 \cdot 6 \cdot 190 \cdot 190 + 15 \cdot 1 \cdot 20 \cdot 1140}{20358520} = \frac{1983600}{20358520} = 0.09743 = 9.743 \%
\end{equation*}
		\item Rješenje zadatka \\
		\\
		Zadatak 13 [0.25 poena] \\
		\\
Podmornica gađa neprijateljski brod sa četiri torpeda, čije su vjerovatnoće
pogađanja 50 \%, 75 \%, 25 \% i 30 \% respektivno. Ako brod pogodi jedan
torpedo, on će biti potopljen sa vjerovatnoćom 35 \%, u slučaju pogotka sa dva
torpeda on će biti potopljen sa vjerovatnoćom 65 \%, dok u slučaju da ga
pogode tri ili četiri torpeda, on se potapa sigurno. Nađite vjerovatnoću da će
brod biti potopljen.
\\
\\
* Vjerovatnoća da će brod biti potopljen jednaka je zbiru vjerovatnoća: \\
\\
1. da ga je pogodio jedan od torpeda $t_i$ pomnožena sa vjerovatnoćom
potapanja ukoliko je pogođen jednim torpedom, \\
2. da su ga pogodila dva od torpeda $t_i$ pomnožena sa vjerovatnoćom potapanja ukoliko je pogođen sa dva torpeda,\\
3. da su ga pogodila tri od torpeda $t_i$ pomnožena sa vjerovatnoćom potapanja ukoliko je pogođen sa tri torpeda,\\
4. da su ga pogodila sva četiri torpeda $t_i$ pomnožena sa vjerovatnoćom potapanja ukoliko je pogođen sa četiri torpeda, odnosno: \\
\begin{equation*}
    P = (0.5 \cdot 0.25 \cdot 0.75 \cdot 0.7 +
         0.5 \cdot 0.75 \cdot 0.75 \cdot 0.7 +
         0.5 \cdot 0.25 \cdot 0.25 \cdot 0.7 +
         0.5 \cdot 0.25 \cdot 0.75 \cdot 0.3) \cdot 0.35
\end{equation*}
\begin{equation*}
     + ~(0.5 \cdot 0.25 \cdot 0.25 \cdot 0.3 +
         0.5 \cdot 0.75 \cdot 0.75 \cdot 0.3 +
         0.5 \cdot 0.75 \cdot 0.25 \cdot 0.7 +~
\end{equation*}
\begin{equation*}
         0.5 \cdot 0.25 \cdot 0.75 \cdot 0.3 +
         0.5 \cdot 0.25 \cdot 0.25 \cdot 0.7 +
         0.5 \cdot 0.75 \cdot 0.75 \cdot 0.7) \cdot 0.65
\end{equation*}
\begin{equation*}
     + ~(0.5 \cdot 0.25 \cdot 0.75 \cdot 0.3 +
         0.5 \cdot 0.25 \cdot 0.25 \cdot 0.7 +
         0.5 \cdot 0.75 \cdot 0.75 \cdot 0.7 +
         0.5 \cdot 0.25 \cdot 0.75 \cdot 0.7) \cdot 1
\end{equation*}
\begin{equation*}
     + ~(0.5 \cdot 0.75 \cdot 0.25 \cdot 0.3) \cdot 1 =
\end{equation*}
\begin{equation*}
     = 0.109375 + 0.2640625 + 0.3125 + 0.028125 = 0.7140625 = 71.40 ~\%
\end{equation*}
		\item Rješenje zadatka \\
		\\
		Zadatak 14 [0.25 poena] \\
		
Za četiri trkača vjerovatnoće da će uspjeti da istrče maraton do kraja
procijenjene su na 55 \%, 75 \%, 50 \% i 40 \% respektivno. Nakon što je maraton
zaista održan, pokazalo se da je samo jedan od njih uspio istrčati maraton do
kraja. Nađite vjerovatnoću da je to bio drugi trkač.
		\newpage
		* Imamo redom vjerovatnoće: \\
		P($T_1$) = 0.55 \\
		P($T_2$) = 0.75 \\
		P($T_3$) = 0.50 \\
		P($T_4$) = 0.40 \\
		Vjerovatnoća da je jedan trkač istrčao je: \\
		\begin{equation*}
		    P(T) = 0.55 \cdot 0.25 \cdot 0.50 \cdot 0.6 + 0.45 \cdot 0.75 \cdot 0.50 \cdot 0.6 
		          + 0.45 \cdot 0.25 \cdot 0.50 \cdot 0.6 + 0.45 \cdot 0.25 \cdot 0.50 \cdot 0.4 =
		\end{equation*}
		\begin{equation*}
		        = 0.19875
		\end{equation*}
		Traži se uslovna vjerovatnoća P($T_2$/T):
		\begin{equation*}
		    P(T_2/T) = \frac{P(T_2) \cdot P(T/T_2)}{P(T)} = \frac{P(T_2) \cdot P(\overline{T_1}) 
		    \cdot  P(\overline{T_3})  \cdot P(\overline{T_4})}{P(T)} = 
		\end{equation*}
		\begin{equation*}
		        = \frac{0.75 \cdot 0.45 \cdot 0.5 \cdot 0.6}{0.19875} = 0.5094 = 50.94 ~\%
		\end{equation*}
		\item Rješenje zadatka \\
		\\
		Zadatak 15 [0.25 poena] \\
		\\
Igrač igra nagradnu igru u kojoj se iz kutije nasumično izvlače kuglice, koje
mogu biti zlatne, srebrene i brončane. Ukupno ima 6 zlatnih, 15 srebrenih i 84
brončanih kuglica. Za osvajanje nagrade potrebno je izvući ili jednu zlatnu
kuglicu (neovisno od toga kakve su ostale), ili dvije srebrene kuglice, ili tri
brončane kuglice. Odredite vjerovatnoću osvajanja nagrade ako igrač ima
pravo izvući \\
a. jednu kuglicu; \\
b. dvije kuglice; \\
c. tri kuglice. \\
\\
		* \\
		a) Ako izvlači jednu kuglicu, jedino može dobiti ako izvuče zlatnu kuglicu,
        znači vjerovatnoću dobijamo kao količnik povoljnih i mogućih slučajeva.
		\begin{equation*}
		    P = \frac{6}{6+15+84} = \frac{6}{105} = 0.0571 = 5.71 ~\%
		\end{equation*}
		b) Ako izvlačimo dvije kuglice, može biti dobitak i sa zlatnim i sa srebrenim kuglicama, pa se vjerovatnoća najlakše računa korištenjem suprotne vjerovatnoće. Slučaji u kojima izvlačimo dvije kuglice, a ne dobijamo su kad izvućemo jednu srebrenu i jednu brozanu ili dvije bronzane, pa se vjerovatnoća računa kao:
		\begin{equation*}
		    P = 1 - \frac{C^2_{84} + C^1_{15} \cdot C^1_{84}}{C^2_{105}}= 1 - \frac{3486 + 1260}{5460} = 0.13 = 13\%
		\end{equation*}
		\newpage
		c)  Ako izvlačimo tri kuglice, ponovo je najlakše vjerovatnoću izračunati
            pomožu suprotne vjerovatnoće, jer je u ovom slučaju jedini scenarij u
            kome igrač ne dobija slučaj kada izvuće dvije bronzane i jednu srebrenu
            kuglicu, pa vjerovatnoću računamo kao:
		\begin{equation*}
		    P = 1 - \frac{C^2_{84} \cdot C^1_{15}}{C^3_{105}}= 1 - \frac{3486 \cdot 15}{187460} = 0.721 = 72.1\%
		\end{equation*}
		\item Rješenje zadatka \\ 
		\\
		Zadatak 16 [0.25 poena] \\
		\\
Računar A je generirao neki binarni podatak (odnosno podatak koji može biti
samo 0 ili 1. Taj podatak je proslijeđen putem lokalne mreže računaru B, koja
je zatim proslijeđena računaru C, i najzad računaru D (sve putem lokalne
mreže). Međutim, uslijed smetnji u prenosu, vjerovatnoća da podatak poslan
sa jednog računara na drugi putem mreže stigne neizmijenjen iznosi svega
49 \%. Greška može uzrokovati da se 0 pretvori u 1 ili 1 pretvori u 0. Ukoliko je
poznato da je na krajnje odredište (računar D) stigao ispravan podatak, kolika
je vjerovatnoća da je na računar B stigao ispravan podatak? \\
\\
* \\
\\
Ako je $A_i$, i ${\in}$ $\{1, 2, 3, 4\}$, označimo kao događaj da neizmijenjen podatak stigne sa prethodnog na sljedeći računar, ta vjerovatnoća može se izraziti kao:
\begin{equation*}
    p(A_1) = p(A_2) = p(A_3) = p(A_4) = 0.49
\end{equation*}
S druge strane, događaj $B_i$, i ${\in}$ $\{1, 2, 3, 4\}$, je događaj da je računar prenio tačnu informaciju. 
Događaji $A_i$ ${\cdot}$ $B_i$ su zavisni te je njihova zavisnost data relacijama:
\begin{equation*}
    B_1 = A_1
\end{equation*}
\begin{equation*}
    B_2 = A_2 \cdot B_1 + \overline{A_2} \cdot \overline{B_1}
\end{equation*}
\begin{equation*}
    B_3 = A_3 \cdot B_2 + \overline{A_3} \cdot \overline{B_2}
\end{equation*}
\begin{equation*}
    B_4 = A_4 \cdot B_3 + \overline{A_4} \cdot \overline{B_3}
\end{equation*}
Pošto ponašanja računara nisu međusobno ovisna, te korištenjem poznate relacije za računanje
suprotne vjerovatnoće dobijamo:
\begin{equation*}
    P(B_1) = P(A_1) = 0.49
\end{equation*}
\begin{equation*}
    P(B_2) = P(A_2) \cdot p(B_1) + P(\overline{A_2}) \cdot P(\overline{B_1}) = 0.49 \cdot 0.49 + 0.51 \cdot 0.51 = 0.5002 
\end{equation*}
\begin{equation*}
    P(B_3) = P(A_3) \cdot P(B_2) + P(\overline{A_3}) \cdot P(\overline{B_2}) = 0.49 \cdot 0.5002 + 0.51 \cdot 0.4998 = 0.499996
\end{equation*}
\begin{equation*}
    P(B_4) = P(A_4) \cdot P(B_3) + P(\overline{A_4}) \cdot P(\overline{B_3}) = 0.49 \cdot 0.499996 + 0.51 \cdot 0.500004 = 0.50000008
\end{equation*}
\newpage
Tražena vjerovatnoća je zapravo $p(B_1/B_4)$ se može izračunati koristeći Bayesovu teoremu, što
daje:
\begin{equation*}
    P(B_1/B_4) = \frac{P(B_1) \cdot P(B_4/B_1)}{P(B_4)}
\end{equation*}
U setu podataka nedostaje još vjerovatnoća $P(B_4/B_1)$ koju su dostupni svi potrebni podaci:
\begin{equation*}
    P(B_2/B_1) = P(A_2) = 0.49
\end{equation*}
\begin{equation*}
    P(B_3/B_1) = P(A_3) \cdot P(B_2/B_1) + P(\overline{A_3}) \cdot P(\overline{B_2}/B_1) = 0.49 \cdot 0.49 + 0.51 \cdot 0.51 = 0.5002 
\end{equation*}
\begin{equation*}
    P(B_4/B_1) = P(A_4) \cdot P(B_3/B_1) + P(\overline{A_4}) \cdot P(\overline{B_3}/B_1) = 0.49 \cdot 0.5002 +
    0.51 \cdot 0.4998 = 0.499996
\end{equation*}
Samim tim tražena vjerovatnoća je:
\begin{equation*}
    P(B_1/B_4) = \frac{P(B_1) \cdot P(B_4/B_1)}{P(B_4)} = \frac{0.49 \cdot 0.499996}{0.50000008} \approx
     49 ~\%
\end{equation*}
	\end{enumerate}
	
	
	
    \end{document}