 \documentclass[12pt]{article}
    \usepackage{amsfonts,amsmath,amssymb}
    \usepackage{amsmath,multicol,eso-pic}
    \usepackage[utf8]{inputenc}
    \usepackage[T1]{fontenc}
    \usepackage[left=2.00cm, right=2.00cm, top=2.00cm, bottom=2.00cm]{geometry}
    \usepackage{titlesec}
    \usepackage{enumerate}
    \usepackage{breqn}
    \usepackage{tikz}
    \usepackage{rotating}
    \usepackage{tikz}
    \usetikzlibrary{automata, positioning}
    %\renewcommand{\wedge}{~}
    %\renewcommand{\neg}{\overline}
    \titleformat{\section}{\large}{\thesection.}{1em}{}
    
    
    % % % % % POPUNITE PODATKE
    
    \newcommand{\prezimeIme}{Mašović Haris}
    \newcommand{\brIndexa}{17993}
    \newcommand{\brZadace}{5}
    
    % % % % % 
    
    \begin{document}
    
    \thispagestyle{empty}
    \begin{center}
      \vspace*{1cm}

      \vspace*{2cm}
      {\huge \bf Zadaća \brZadace } \\
      \vspace*{1cm}
      {\Large \bf iz predmeta Diskretna Matematika}

      \vspace*{1.25cm}

      {\Large Prezime i ime: \prezimeIme} \\
      \vspace*{0.5cm}
      {\Large Br. indexa: \brIndexa} \\
      \vspace*{0.5cm}
      {\Large Demonstrator: Rijad Muminović} \\
      \vspace*{0.5cm}
      {\Large Grupa: RI - 4} \\ 
      
      \vspace*{2cm}
      \renewcommand{\arraystretch}{1.75}
      \begin{tabular}{|c|c|}
    	\hline Zadatak & Bodovi \\
    	\hline 1 &  \\
    	\hline 2 &  \\
    	\hline 3 &  \\
    	\hline 4 &  \\
    	\hline 5 &  \\
    	\hline 6 &  \\
    	\hline 7 &  \\
    	\hline 8 &  \\
    	\hline
     \end{tabular}

      \vfill


      {\large Elektrotehnički fakultet Sarajevo, Januar 2018g.}

    \end{center}
    \newpage
    \thispagestyle{empty}
    
    
    % % % % % Rješenja zadataka
	\begin{enumerate}
		\item Rješenje zadatka \\
		\\
		Zadatak 1 [0.5 poena] \\
		\\
		a) Predstavite ovaj diskretni signal formulom u kojoj se javlja cijeli dio broja.
		\\
		\\
		Svaki periodičan signal perioda N može se izraziti formulom u kojoj figurira cijeli dio broja na sljedeći način: \\
		\begin{equation*}
		    x_n = x_{-1} + \sum_{k =0}^{N-1}~(x_k - x_{k-1}) \lfloor~\frac{n - k }{N}~\rfloor
		\end{equation*}
		U zadanom primjeru poznate su vrijednosti $x_n$ a n = [0, 5] i one redom iznose -9, -6, 2, -2, 5 i -2. Zbog periodičnosti, poznata je i vrijednost $x_{-1}$ = $x_5$ = -2. Uvrštavanjem ovih vrijednosti u navedenu formulu dobija se sljedeći izraz:
		\begin{equation*}
		    x_n = -2 -7 \cdot \lfloor~\frac{n }{6}~\rfloor + 3 \cdot \lfloor~\frac{n - 1}{6}~\rfloor + 8 \cdot \lfloor~\frac{n - 2}{6}~\rfloor - 4 \cdot \lfloor~\frac{n - 3}{6}~\rfloor + 7 \cdot \lfloor~\frac{n - 4}{6}~\rfloor - 7 \cdot \lfloor~\frac{n - 5}{6}~\rfloor
		\end{equation*}
		\\
		b) Predstavite ovaj signal diskretnim Fourierovim redom.
		\\
		\\
		Svaki periodični signal sa periodom N također je moguće izraziti preko diskretnog Fourierovog reda. Njegov oblik i koeficijenti dati su formulama:
		\begin{equation*}
		    x_n = \frac{a_0}{2} + \sum_{k = 1}^{\lfloor~N/2~\rfloor} a_k~cos\frac{2k\pi}{N}n + b_k~sin\frac{2k\pi}{N}n
		\end{equation*}
		\begin{equation*}
		    a_k = \frac{2}{N}~\sum_{n = 0}^{N - 1} x_n~cos\frac{2k\pi}{N}n, ~ k = 0,1,...,\lfloor~\frac{N}{2}~\rfloor
		\end{equation*}
		\begin{equation*}
		    b_k = \frac{2}{N}~\sum_{n = 0}^{N - 1} x_n~sin\frac{2k\pi}{N}n, ~ k = 0,1,...,\lfloor~\frac{N}{2}~\rfloor
		\end{equation*}
		Iznimku pri upotrebi ove formule predstavlja situacija kada je N paran broj, jer se tada pri računanju koeficijenta $a_k$ za k = N/2 ispred sume umjesto faktora 2/N javlja faktor 1/N. Također, $b_k$ za takvo k jednako nuli za parno N. Potrebni koeficijenti su:
		\begin{equation*}
		    a_0 = -4 \quad a_1 = \frac{-29}{6} \quad a_2 = \frac{-7}{2} \quad a_3 = \frac{8}{3}
		\end{equation*}
		\begin{equation*}
		    b_1 = \frac{-7 \cdot \sqrt{3}}{6} \quad  b_2 = \frac{-\sqrt{3}}{6} \quad b_3 = 0
		\end{equation*}
		Diskretni Fourierov red glasi:
		\begin{equation*}
		    x_n = -2 + \frac{-29}{6}~cos\frac{\pi}{3}n + \frac{-7 \cdot \sqrt{3}}{6}~sin\frac{\pi}{3}n + \frac{-7}{2}~cos\frac{2\pi}{3}n + \frac{-\sqrt{3}}{6}~sin\frac{2\pi}{3}n + \frac{8}{3}~cos\pi n
		\end{equation*}
		\newpage
		c) Odredite amplitudni i fazni spektar za ovaj signal i predstavite ga u vidu sume harmonika.
		\begin{equation*}
		    x_n = \sum_{k = 0}^{\lfloor~N/2~\rfloor} A_k~sin(\frac{2k\pi}{N}n + \varphi_k) ~(*)
		\end{equation*}
		\begin{equation*}
		    A_k = \sqrt{a_k^2 + b_k^2} \quad \varphi_k = arctg\frac{a_k}{b_k}
		\end{equation*}
		pa imamo:
		\begin{equation*}
		    A_0 = 4 \quad A_1 = \frac{\sqrt{247}}{3} \quad A_2 = \frac{\sqrt{111}}{3}\quad A_3 = \frac{8}{3}
		\end{equation*}
		\begin{equation*}
		    \varphi_0 = \frac{3\pi}{2} \quad \varphi_1 = arctg\frac{29}{7\sqrt{3}}\quad \varphi_2 = arctg\frac{21}{\sqrt{3}}\quad \varphi_3 = \frac{\pi}{2}
		\end{equation*}
		\\
		Uvrštavanje vrijednosti u (*) imamo:
		\begin{equation*}
		    x_n = 4~sin(\frac{3\pi}{2}) + \frac{\sqrt{247}}{3}~sin(\frac{\pi}{3}n + arctg\frac{29}{7\sqrt{3}}) + \frac{\sqrt{111}}{3}~sin(\frac{2\pi}{3}n + arctg\frac{21}{\sqrt{3}}) + \frac{8}{3}~sin(\pi n + \frac{\pi}{2})
		\end{equation*}
		\item Rješenje zadatka \\
		\\
		Zadatak 2 [0.4 poena] \\
		\\
\end{enumerate}
\end{document}