%MIT License
%
%Copyright (c) 2017 dinkoosmankovic
%
%Permission is hereby granted, free of charge, to any person obtaining a copy
%of this software and associated documentation files (the "Software"), to deal
%in the Software without restriction, including without limitation the rights
%to use, copy, modify, merge, publish, distribute, sublicense, and/or sell
%copies of the Software, and to permit persons to whom the Software is
%furnished to do so, subject to the following conditions:
%
%The above copyright notice and this permission notice shall be included in all
%copies or substantial portions of the Software.
%
%THE SOFTWARE IS PROVIDED "AS IS", WITHOUT WARRANTY OF ANY KIND, EXPRESS OR
%IMPLIED, INCLUDING BUT NOT LIMITED TO THE WARRANTIES OF MERCHANTABILITY,
%FITNESS FOR A PARTICULAR PURPOSE AND NONINFRINGEMENT. IN NO EVENT SHALL THE
%AUTHORS OR COPYRIGHT HOLDERS BE LIABLE FOR ANY CLAIM, DAMAGES OR OTHER
%LIABILITY, WHETHER IN AN ACTION OF CONTRACT, TORT OR OTHERWISE, ARISING FROM,
%OUT OF OR IN CONNECTION WITH THE SOFTWARE OR THE USE OR OTHER DEALINGS IN THE
%SOFTWARE.


    \documentclass[12pt]{article}
    \usepackage{amsfonts,amsmath,amssymb}
    \usepackage{amsmath,multicol,eso-pic}
    \usepackage[utf8]{inputenc}
    \usepackage[T1]{fontenc}
    \usepackage[left=2.00cm, right=2.00cm, top=2.00cm, bottom=2.00cm]{geometry}
    \usepackage{titlesec}
    \usepackage{enumerate}
    \usepackage{breqn}
    \usepackage{tikz}
    \usepackage{rotating}
    %\renewcommand{\wedge}{~}
    %\renewcommand{\neg}{\overline}
    \titleformat{\section}{\large}{\thesection.}{1em}{}
    
    
    % % % % % POPUNITE PODATKE
    
    \newcommand{\prezimeIme}{Mašović Haris}
    \newcommand{\brIndexa}{17993}
    \newcommand{\brZadace}{4}
    
    % % % % % 
    
    \begin{document}
    
    \thispagestyle{empty}
    \begin{center}
      \vspace*{1cm}

      \vspace*{2cm}
      {\huge \bf Zadaća \brZadace } \\
      \vspace*{1cm}
      {\Large \bf iz predmeta Matematička logika i teorija izračunljivosti}

      \vspace*{2cm}

      {\Large Prezime i ime: \prezimeIme} \\
      \vspace*{0.75cm}
      {\Large Br. indexa: \brIndexa}

      \vspace*{3cm}
      \renewcommand{\arraystretch}{1.75}
      \begin{tabular}{|c|c|}
    	\hline Zadatak & Bodovi \\
    	\hline 1 &  \\
    	\hline 2 &  \\
    	\hline 3 &  \\
    	\hline 4 &  \\
    	\hline 5 &  \\
    	\hline 6 &  \\
    	\hline 7 &  \\
    	\hline 
     \end{tabular}

      \vfill


      {\large Elektrotehnički fakultet Sarajevo}

    \end{center}
    \newpage
    \thispagestyle{empty}
    
    
    % % % % % Rješenja zadataka
	\begin{enumerate}
		\item Rješenje zadatka
		
		* Ukoliko označimo date jezike sa: \\
		S - Španski \\
		I - Italijanski \\
		P - Portugalski \\
		J - Japanski \\
		\\
		* Iz postavke imamo sljedeće uslove: \\
		\#(S {$\cup$} I {$\cup$} P {$\cup$} J) = 277 \\
		\#(S {$\cap$} I) = 23  \\
		\#(S {$\cap$} P) = 23  \\
		\#(S {$\cap$} J) = 23  \\
		\#(I {$\cap$} P) = 23  \\
	    \#(I {$\cap$} J) = 23  \\ 
		\#(P {$\cap$} J) = 23  \\
		\#(S {$\cap$} I {$\cap$} P) = 11 \\
		\#(S {$\cap$} P {$\cap$} J) = 11 \\
		\#(S {$\cap$} I {$\cap$} J) = 11 \\
		\#(I {$\cap$} P {$\cap$} J) = 11 \\
		\#(S {$\cap$} I {$\cap$} P {$\cap$} J) = 5 \\
		\#S = \#I + \#P ~~ (1)\\
		\#I = \#J + 4 ~~ (2)\\
		\#P = \#J - 6 ~~ (3)\\
		
		* Dalje možemo izračunati kardinalni broj svakog skupa na sljedeći način i \\ korštenje jednačina (1),(2),(3): \\
		
		\#(S {$\cup$} I {$\cup$} P {$\cup$} J) = 277 = \#S + \#I + \#P + \#J -(\#(S {$\cap$} I)+ \#(S {$\cap$} P)+ \#(S {$\cap$} J)+ \\\#(I {$\cap$} P)+ \#(I {$\cap$} J)+ \#(P {$\cap$} J)) + \#(S {$\cap$} I {$\cap$} P) + \#(S {$\cap$} P {$\cap$} J) + \#(S {$\cap$} I {$\cap$} J) + \\\#(I {$\cap$} P {$\cap$} J) - \#(S {$\cap$} I {$\cap$} P {$\cap$} J)  = \\
		384 = 5\#J - 4 {$\Rightarrow$} \#J = 76 \\
		\\
		** Uvrštavajući nazad u uslove (1),(2),(3) imamo: \\
		\#P = 70, \#I = 80, \#S = 150 \\
		\item Rješenje zadatka
		\\
		a) 
		\begin{equation*}
		
		   (A \setminus B) \setminus C = A \setminus ( B \cup C) \\
		   (A \cap C(B)) \setminus C = A \cap C(B \cup C) \\ 
		   A \cap C(B) \cap C(C) = A \cap C(B) \cap C(C) \\		   
		\end{equation*}
		\\
		
		b)
		\begin{equation*}
		
		   (A \setminus B) \ (B \setminus C) = A \setminus B \\
           (A  \cap C(B)) \setminus (B \cap C(C)) = A \cap C(B) \\ 
		   A \cap C(B) \cap (C(B) \cup C) = A \cap C(B) \\
		   A \cap (C(B) \cup (C(B) \cap C)) = A  \cap C(B) \\
		   A \cap (C(B) \cap (C \cup U)) = A \cap C(B) \\
		   A \cap C(B) = A \cap C(B) \\
		\end{equation*}
		\\		
		\item Rješenje zadatka
		
		a) \\
		
		* Ovo ćemo dokazati pomoću zakona idempotentnosti koji glasi: \\
		A {$\cup$} A {$\cup$} A {$\cup$} ... {$\cup$} A = A \\
		
		* Dalje ukoliko raspišemo ovih n skupova imamo: \\
		A {$\cap$} ($B_1$ {$\cup$} $B_2$ {$\cup$} $B_3$ {$\cup$} ... {$\cup$} $B_n$) = (A {$\cap$} $B_1$) {$\cup$} (A {$\cap$} $B_2$) {$\cup$} (A {$\cap$} $B_3$) {$\cup$} ... {$\cup$} (A {$\cap$} $B_n$) \\
		
		** Isključenjem trećeg imamo: \\
		
		A {$\cap$} ($B_1$ {$\cup$} $B_2$ {$\cup$} $B_3$ {$\cup$} ... {$\cup$} $B_n$) = A {$\cap$} ($B_1$ {$\cup$} $B_2$ {$\cup$} $B_3$ {$\cup$} ... {$\cup$} $B_n$) \\
		
		*** Samim tim dokaz vrijedi uz početne uslove.
		
		b) \\
		
		* Ovo ćemo dokazati pomoću zakona idempotentnosti koji glasi: \\
		A {$\cap$} A {$\cap$} A {$\cap$} ... {$\cap$} A = A \\
		
		* Dalje ukoliko raspišemo ovih n skupova imamo: \\
		(A x $B_1$) {$\cap$} (A x $B_2$) {$\cap$} (A x $B_3$) {$\cap$} ... {$\cap$} (A x $B_n$) = A x ($B_1$ {$\cap$} $B_1$ {$\cap$} $B_1$ {$\cap$} ... {$\cap$} $B_n$) \\
		
		** Isključenjem trećeg imamo: \\
		
		A x ($B_1$ {$\cap$} $B_1$ {$\cap$} $B_1$ {$\cap$} ... {$\cap$} $B_n$) = A x ($B_1$ {$\cap$} $B_1$ {$\cap$} $B_1$ {$\cap$} ... {$\cap$} $B_n$) \\
		
		*** Samim tim dokaz vrijedi uz početne uslove. 
		\newpage
		\item Rješenje zadatka \\
		
		* Funkcija očigledno nije inverzna jer nije zadovoljeno pravilo '1-1' tj. injektivnost , zatim funkcija nije surjektivna jer kodomen nije čitav Z, što znači da funkcija nije ni bijektivna samim tim nema inverznu funkciju. \\
		
		** Generalizirana inverzna funkcija glasi: \\
		
		$F^{-1}(x) =$
        \begin{cases}
        \{\emptyset\}~~~~~~~~~~~~~~~~~~~~~~~~~~~,~~x=0\\
        \{\{a\},\{b\},\{c\}\}~~~~~~~~~~~~~,~~x=1\\
        \{\{a,b\},\{a,c\},\{b,c\}\}~~~~~,~~x = 2\\
        \{a,b,c\}~~~~~~~~~~~~~~~~~~~~~~~,~~x = 3\\
        \emptyset~~~~~~~~~~~~~~~~~~~~~~~~~~~~~~~,~~$x~\in\mathbb{Z}$ \textbackslash~ \{0,1,2,3\} \\
        \end{cases}
		
		\\
		\\
		\\
		\\
		\\
		
		\item Rješenje zadatka
		
		$$((\lambda f.\lambda x.f(f(x)))((\lambda y.2y + 1)
		(\lambda t.t^3 + 2t^2 - t + 4)))(2) =$$
		$$= ((\lambda f.\lambda x. f(f(x)))(2(\lambda t. t^3 +
		2t^2 - t + 4) + 1))(2) =$$
		$$\triangleright f = 2(\lambda t.t^3 + 2t^2 - t + 4) +
		1$$
		$$= (\lambda x.f(f(x)))(2) =$$
		$$= f(f(2)) =$$
		$$= f((2(\lambda t.t^3 + 2t^2 - t + 4) + 1)(2)) =$$
		$$= f(2(2^3 + 2\cdot 2^2 - 2 + 4) + 1) =$$
		$$= f(37) =$$
		$$= (2(\lambda t.t^3 + 2t^2 - t + 4) + 1)(37) =$$
		$$= 2(37^3 + 2\cdot 37^2 - 37 + 4) + 1 =$$
		$$= 2(50653 + 2738 - 37 + 4) + 1 =$$
		$$= 106717$$
		
		
		
		\newpage
		\item Rješenje zadatka
		
		a) \\
		
		Vrijedi $xR^{-1}y$ ako i samo ako 
		 	vrijedi $yRx$.
		 	$$R^{-1} = \{(6, 1), (5, 2), (5, 3), (5, 4), (4, 5)
		 	, (5, 5)\}$$
		 	Relaciju R možemo predstaviti relacionom matricom:
		 	\begin{equation*}
				M = 
				\begin{pmatrix} 
					0 & 0 & 0 & 0 & 0 & 1\\ 
					0 & 0 & 0 & 0 & 1 & 0\\
					0 & 0 & 0 & 0 & 1 & 0\\
					0 & 0 & 0 & 0 & 1 & 0\\
					0 & 0 & 0 & 1 & 1 & 0\\
					0 & 0 & 0 & 0 & 0 & 0\\
				\end{pmatrix}
			\end{equation*}
		 	Sada relaciju $R^2$ i $R^3$ možemo naci pomocu
		 	Booleovog proizvoda matrica:
		 	\begin{equation*}
				M^2 = 
				\begin{pmatrix} 
					0 & 0 & 0 & 0 & 0 & 1\\ 
					0 & 0 & 0 & 0 & 1 & 0\\
					0 & 0 & 0 & 0 & 1 & 0\\
					0 & 0 & 0 & 0 & 1 & 0\\
					0 & 0 & 0 & 1 & 1 & 0\\
					0 & 0 & 0 & 0 & 0 & 0\\
				\end{pmatrix} \circ
				\begin{pmatrix} 
					0 & 0 & 0 & 0 & 0 & 1\\ 
					0 & 0 & 0 & 0 & 1 & 0\\
					0 & 0 & 0 & 0 & 1 & 0\\
					0 & 0 & 0 & 0 & 1 & 0\\
					0 & 0 & 0 & 1 & 1 & 0\\
					0 & 0 & 0 & 0 & 0 & 0\\
				\end{pmatrix} =
				\begin{pmatrix} 
					0 & 0 & 0 & 0 & 0 & 0\\ 
					0 & 0 & 0 & 1 & 1 & 0\\
					0 & 0 & 0 & 1 & 1 & 0\\
					0 & 0 & 0 & 1 & 1 & 0\\
					0 & 0 & 0 & 1 & 1 & 0\\
					0 & 0 & 0 & 0 & 0 & 0\\
				\end{pmatrix}
			\end{equation*}
		 	$$R^2 = \{(2, 4), (2, 5), (3, 4), (3, 5), (4, 4), 
		 	(4, 5), (5, 4), (5, 5)\}$$
		 	\begin{equation*}
		 	\hspace{-2cm} M^3 = M^2 \circ M = 
		 	\begin{pmatrix} 
					0 & 0 & 0 & 0 & 0 & 0\\ 
					0 & 0 & 0 & 1 & 1 & 0\\
					0 & 0 & 0 & 1 & 1 & 0\\
					0 & 0 & 0 & 1 & 1 & 0\\
					0 & 0 & 0 & 1 & 1 & 0\\
					0 & 0 & 0 & 0 & 0 & 0\\
				\end{pmatrix} \circ
				\begin{pmatrix} 
					0 & 0 & 0 & 0 & 0 & 1\\ 
					0 & 0 & 0 & 0 & 1 & 0\\
					0 & 0 & 0 & 0 & 1 & 0\\
					0 & 0 & 0 & 0 & 1 & 0\\
					0 & 0 & 0 & 1 & 1 & 0\\
					0 & 0 & 0 & 0 & 0 & 0\\
				\end{pmatrix} = 
				\begin{pmatrix} 
					0 & 0 & 0 & 0 & 0 & 0\\ 
					0 & 0 & 0 & 1 & 1 & 0\\
					0 & 0 & 0 & 1 & 1 & 0\\
					0 & 0 & 0 & 1 & 1 & 0\\
					0 & 0 & 0 & 1 & 1 & 0\\
					0 & 0 & 0 & 0 & 0 & 0\\
				\end{pmatrix}
		 	\end{equation*}
		 	$$R^3 = R^2 = \{(2, 4), (2, 5), (3, 4), (3,
		 	5), (4, 4), (4, 5), (5, 4), (5, 5)\}$$
		 	Simetricno zatvorenje jeste unija relacije i njene
		 	inverzne relacije.
		 	$$R^s = R \cup R^{-1} = \{(1,6),(2,5),(3,5),(4, 5),
		 	(5, 2),(5, 3), (5,4), (5,5), (6, 1)\}$$
		 	Vrijedi $R^+$ ako i samo ako je ikako moguce iz x
		 	doci u y na strelicastom dijagramu tj. sve vrijednosti {$\top$} bez glavne dijagonale.
		 	$$R^+ = \{(1, 6), (2, 4), (2, 5), (3, 4), (3, 5),
		 	(4, 4), (4, 5), (5, 4), (5, 5)\}$$
		 	Da bismo našli $R^*$, zatvorenje $R^+$
		 	dopunjujemo sa refleksivnošću.
		 	$$R^* = \{(1, 1), (1, 6), (2, 2), (2, 4), (2, 5),
		 	(3,3), (3, 4), (3, 5), (4, 4), (4, 5), (5, 4),(5, 5), (6, 6), (7, 7)\}$$
		 			 	\newpage
		 	b) \\
		 	\begin{equation*}
				M \circ M^{-1} = 
				\begin{pmatrix} 
					0 & 0 & 0 & 0 & 0 & 1\\ 
					0 & 0 & 0 & 0 & 1 & 0\\
					0 & 0 & 0 & 0 & 1 & 0\\
					0 & 0 & 0 & 0 & 1 & 0\\
					0 & 0 & 0 & 1 & 1 & 0\\
					0 & 0 & 0 & 0 & 0 & 0\\
				\end{pmatrix} \circ 
				\begin{pmatrix} 
					0 & 0 & 0 & 0 & 0 & 0\\ 
					0 & 0 & 0 & 0 & 0 & 0\\
					0 & 0 & 0 & 0 & 0 & 0\\
					0 & 0 & 0 & 0 & 1 & 0\\
					0 & 1 & 1 & 1 & 1 & 0\\
					1 & 0 & 0 & 0 & 0 & 0\\
				\end{pmatrix} = 
				\begin{pmatrix} 
					1 & 0 & 0 & 0 & 0 & 0\\ 
					0 & 1 & 1 & 1 & 1 & 0\\
					0 & 1 & 1 & 1 & 1 & 0\\
					0 & 1 & 1 & 1 & 1 & 0\\
					0 & 1 & 1 & 1 & 1 & 0\\
					0 & 0 & 0 & 0 & 0 & 0\\
				\end{pmatrix}				
			\end{equation*}
			$$R \circ R^{-1} = \{(1, 1), (2, 2), (2, 3), (2, 4)
			, (2, 5), (3, 2), (3, 3), (3, 4), (3, 5), (4, 2), 
			(4, 3), (4, 4),$$
			$$(4, 5), (5, 2), (5, 3), (5, 4), (5, 5)\}$$
			\begin{equation*}
				M^2 \circ (M^{-1})^2 =
				\begin{pmatrix} 
					0 & 0 & 0 & 0 & 0 & 0\\ 
					0 & 0 & 0 & 1 & 1 & 0\\
					0 & 0 & 0 & 1 & 1 & 0\\
					0 & 0 & 0 & 1 & 1 & 0\\
					0 & 0 & 0 & 1 & 1 & 0\\
					0 & 0 & 0 & 0 & 0 & 0\\
				\end{pmatrix} \circ
				\begin{pmatrix} 
					0 & 0 & 0 & 0 & 0 & 0\\ 
					0 & 0 & 0 & 0 & 0 & 0\\
					0 & 0 & 0 & 0 & 0 & 0\\
					0 & 1 & 1 & 1 & 1 & 0\\
					0 & 1 & 1 & 1 & 1 & 0\\
					0 & 0 & 0 & 0 & 0 & 0\\
				\end{pmatrix} =
				\begin{pmatrix} 
					0 & 0 & 0 & 0 & 0 & 0\\ 
					0 & 1 & 1 & 1 & 1 & 0\\
					0 & 1 & 1 & 1 & 1 & 0\\
					0 & 1 & 1 & 1 & 1 & 0\\
					0 & 1 & 1 & 1 & 1 & 0\\
					0 & 0 & 0 & 0 & 0 & 0\\
				\end{pmatrix}
			\end{equation*}
			$$R^2 \circ (R^{-1})^2 = \{(2, 2), (2, 3), (2, 4)
			, (2, 5), (3, 2), (3, 3), (3, 4), (3, 5), (4, 2), 
			(4, 3), (4, 4),$$
			$$ (4, 5), (5, 2), (5, 3), (5, 4), (5, 5)\}$$

		 c)
		 
		    \begin{equation*}
				(M \circ M^2) \circ M^3 =
				\resizebox{0.2cm}{1cm}{$($}
				\begin{pmatrix} 
					0 & 0 & 0 & 0 & 0 & 1\\ 
					0 & 0 & 0 & 0 & 1 & 0\\
					0 & 0 & 0 & 0 & 1 & 0\\
					0 & 0 & 0 & 0 & 1 & 0\\
					0 & 0 & 0 & 1 & 1 & 0\\
					0 & 0 & 0 & 0 & 0 & 0\\
				\end{pmatrix} \circ
				\begin{pmatrix} 
					0 & 0 & 0 & 0 & 0 & 0\\ 
					0 & 0 & 0 & 1 & 1 & 0\\
					0 & 0 & 0 & 1 & 1 & 0\\
					0 & 0 & 0 & 1 & 1 & 0\\
					0 & 0 & 0 & 1 & 1 & 0\\
					0 & 0 & 0 & 0 & 0 & 0\\
				\end{pmatrix}
				\resizebox{0.2cm}{1cm}{$)$} \circ
				\begin{pmatrix} 
					0 & 0 & 0 & 0 & 0 & 0\\ 
					0 & 0 & 0 & 1 & 1 & 0\\
					0 & 0 & 0 & 1 & 1 & 0\\
					0 & 0 & 0 & 1 & 1 & 0\\
					0 & 0 & 0 & 1 & 1 & 0\\
					0 & 0 & 0 & 0 & 0 & 0\\
				\end{pmatrix} = 
			\end{equation*}
			\begin{equation*}
			\hspace{3cm}	\begin{pmatrix} 
					0 & 0 & 0 & 0 & 0 & 0\\ 
					0 & 0 & 0 & 1 & 1 & 0\\
					0 & 0 & 0 & 1 & 1 & 0\\
					0 & 0 & 0 & 1 & 1 & 0\\
					0 & 0 & 0 & 1 & 1 & 0\\
					0 & 0 & 0 & 0 & 0 & 0\\
				\end{pmatrix} \circ
				\begin{pmatrix} 
					0 & 0 & 0 & 0 & 0 & 0\\ 
					0 & 0 & 0 & 1 & 1 & 0\\
					0 & 0 & 0 & 1 & 1 & 0\\
					0 & 0 & 0 & 1 & 1 & 0\\
					0 & 0 & 0 & 1 & 1 & 0\\
					0 & 0 & 0 & 0 & 0 & 0\\
				\end{pmatrix} =
				\begin{pmatrix} 
					0 & 0 & 0 & 0 & 0 & 0\\ 
					0 & 0 & 0 & 1 & 1 & 0\\
					0 & 0 & 0 & 1 & 1 & 0\\
					0 & 0 & 0 & 1 & 1 & 0\\
					0 & 0 & 0 & 1 & 1 & 0\\
					0 & 0 & 0 & 0 & 0 & 0\\
				\end{pmatrix}
				\end{equation*}
				\\
				
				$$(R \circ R^2) \circ R^3 = \{(2, 4), (2, 5),
				(3, 4), (3, 5), (4, 4), (4, 5), (5, 4), 
				(5, 5)\}$$
		\newpage
		\item Rješenje zadatka
		
		* Ukoliko posmatramo našu prvu projekciju relacije R2 imamo: \\
		
		\begin{equation*}
    		    \underset{(\#1,\#2)}{\pi}(R2) = \{(8,6),(7,4),(6,5),(2,6),(4,8)\}
		\end{equation*}
		
		* Dalje ukoliko posmatramo lambda spajanje, vidimo da moramo spojiti petorku sa dvojkom, pošto u narednoj iteraciji su nam potrebne samo dvije dobivene sedmorke samo će one biti napisane, tj. idemo za korak više pa imamo: \\
		
		\begin{equation*}
    		    \text{A} = \underset{(\#1>\#2~\wedge~\#1 \neg 1)}{\sigma}~( R1 ~ \underset{(\#1<\#1+\#2)}{\bowtie}~\underset{(\#1,\#2)}{\pi}(R2)) = \{(5,4,3,2,6,8,6),(5,4,3,2,6,7,4), \\
    		    
    		    (5,4,3,2,6,6,5),(5,4,3,2,6,2,6),(5,4,3,2,6,4,8), (4,1,1,5,7,8,6),(4,1,1,5,7,7,4),(4,1,1,5,7,6,5), \\
    		    
    		    (4,1,1,5,7,2,6),(4,1,1,5,7,4,8)\}
		\end{equation*}
		
		** Na kraju posmatramo projekciju i dobijamo konačnu relaciju R: \\
		
		\begin{equation*}
		    R = \underset{(\#1,\#2,\#(-1))}{\pi}(A) = \{ (5,4,6),(5,4,4),(5,4,5),(5,4,6),
		    (5,4,8),(4,1,6),(4,1,4),\\
		    
		    (4,1,5),(4,1,8)\}
		\end{equation*}
		
	\end{enumerate}
	
	
	
    \end{document}
    